\documentclass[11pt]{article}
\usepackage{setspace}
\setstretch{1}
\usepackage{amsmath,amssymb, amsthm}
\usepackage{graphicx}
\usepackage{bm}
\usepackage[hang, flushmargin]{footmisc}
\usepackage[colorlinks=true]{hyperref}
\usepackage[nameinlink]{cleveref}
\usepackage{footnotebackref}
\usepackage{url}
\usepackage{listings}
\usepackage[most]{tcolorbox}
\usepackage{inconsolata}
\usepackage[papersize={8.5in,11in}, margin=1in]{geometry}
\usepackage{float}
\usepackage{caption}
\usepackage{esint}
\usepackage{url}
\usepackage{enumitem}
\usepackage{subfig}
\usepackage{wasysym}
\newcommand{\ilc}{\texttt}
\usepackage{etoolbox}
\usepackage{algorithm}
\usepackage{changepage}
% \usepackage{algorithmic}
\usepackage[noend]{algpseudocode}
\usepackage{tikz}
\usetikzlibrary{matrix,positioning,arrows.meta,arrows}
\patchcmd{\thebibliography}{\section*{\refname}}{}{}{}
% \PassOptionsToPackage{hyphens}{url}\usepackage{hyperref}

\providecommand{\myceil}[1]{\left \lceil #1 \right \rceil }
\providecommand{\myfloor}[1]{\left \lfloor #1 \right \rfloor }


\begin{document}



\title{\textbf{MATH 307: Individual Homework 4}}


\author{Shaochen (Henry) ZHONG, \ilc{sxz517@case.edu}}

\date{Due and submitted on 02/17/2021 \\ Spring 2021, Dr. Guo}
\maketitle

\subsection*{Problem 1}
\textit{Textbook page 40, problem 1.}\newline

For $W$ to be a vector space, we must have $(u + v) \in W$ for $u, v \in W$. However for $u = v = \begin{bmatrix} 2\\ 2\\ 2\\ 2\\\end{bmatrix}$, we have $(u + v) = \begin{bmatrix} 4\\ 4\\ 4\\ 4\\ \end{bmatrix}$; which is $\not\in W$ as $|x_j| \not< 3$ for $i \leq j \leq 4$. Thus, we may conclude that $W$ is not a vector space.


\subsection*{Problem 2}
\textit{Textbook page 40, problem 5.}\newline

For $W \in \mathbb{R}^3: x + 20y - 12z - 1 = 0$ for $x, y, z$ as elements of $W$ to be a vector space, we must have a zero vector $0$ where $0 + u = u$ for $u \in W$. In this case the zero vector is $\begin{bmatrix} 0\\ 0\\ 0\\ \end{bmatrix}$ but it is $\not \in W$ as $0 + 20(0) - 12(0) - 1 \neq 0$. Thus, we may conclude that $W$ is not a vector space.


\subsection*{Problem 3}
\textit{See HW instruction.}\newline

For matrices $M, N, U \in F^{n \times n}$, we know that $tr(M) = tr(N) = tr(U) = 0$. Since the trace of a matrix is only about its diagonal elements, lets assume we have the $tr(M) = M_{11} + M_{22} + ... +  M_{nn} = 0$, $tr(N) = N_{11} + N_{22} + ... +  N_{nn} = 0$, and same for $tr(U)$; where the subscript is the index of element. Also assume we have scalar $\lambda, \mu \in F$, We have:

\begin{enumerate}
    \item $M + N \in  F^{n \times n}$ as $tr(M + N) = (M_{11} + N_{11}) + (M_{22} + N_{22}) +  ... + (M_{nn} + N_{nn}) = 0$.
    \item $M + N = N + M = \begin{bmatrix}
        M_{11} + N_{11} & M_{12} + N_{12} & \cdots &M_{1n} + N_{1n} \\
        M_{21} + N_{21} & M_{22} + N_{22} & \cdots &M_{2n} + N_{2n} \\
        \vdots & \vdots & \ddots & \vdots \\
        M_{n1} + N_{n1} & M_{n2} + N_{n2} & \cdots &M_{nn} + N_{nn} \\
    \end{bmatrix}$
    \item $U + (M + N) = (U + M) + N = \begin{bmatrix}
        M_{11} + N_{11}+ U_{11} & M_{12} + N_{12}+ U_{12} & \cdots & M_{1n} + N_{1n}+ U_{1n} \\
        M_{21} + N_{21} + U_{21} & M_{22} + N_{22} + U_{22} & \cdots&  M_{2n} + N_{2n} + U_{2n} \\
        \vdots & \vdots & \ddots & \vdots \\
        M_{n1} + N_{n1}+U_{n1} & M_{n2} + N_{n2}+U_{n2} & \cdots & M_{nn} + N_{nn}+U_{nn} \\
    \end{bmatrix}$
    \item $0 \in  F^{n \times n}$ where $0 + M = M$. We know this is true as as each element on $ij$ index in LHS is $M_{ij}$ which is equals to the $ij$-indexed element in RHS. We also know that a $n \times n$ matrix filled with $0$s is $\in  F^{n \times n}$.
    \item $\forall M \in F^{n \times n}$, we have $M + -M = 0$ where the element on $ij$ index in $-M$ is simply $-1 \cdot M_{ij}$, so we must have $M_{ij} + (-M_{ij}) = 0$ and therefore $M + -M = 0$.
    \item $\lambda M \in F^{n \times n}$ as the element on $ij$ index in $\lambda M$ is simply $\lambda \cdot M_{ij}$ which is still in $F$, we then have $\lambda M  \in F^{n \times n}$.
    \item $\lambda (M + N) = \lambda M + \lambda N$ as each element on $ij$ index in LHS is $\lambda(M_{ij} + N_{ij}) = \lambda M_{ij} + \lambda N_{ij}$, which is equals to the $ij$-indexed element in RHS.
    \item $(\lambda + \mu) M = \lambda M + \mu M$ as each element on $ij$ index in LHS is $(\lambda + \mu) M_{ij} =\lambda M_{ij} + \mu M_{ij}$, which is equals to the $ij$-indexed element in RHS.
    \item $\lambda (\mu M) = (\lambda \mu) M$ as each element on $ij$ index in LHS is $\lambda \cdot \mu \cdot  M_{ij}$ which is equals to the $ij$-indexed element in RHS.
    \item $1 \cdot M = M$  as each element on $ij$ index in LHS is $1 \cdot M_{ij}$ which is equals to the $ij$-indexed element in RHS.
\end{enumerate}\newline

As all ten axioms are proven to be valid, we may conclude that $F^{n \times n}$ is vector field over $F$.






\end{document}


\documentclass[11pt]{article}
\usepackage{setspace}
\setstretch{1}
\usepackage{amsmath,amssymb, amsthm}
\usepackage{graphicx}
\usepackage{bm}
\usepackage[hang, flushmargin]{footmisc}
\usepackage[colorlinks=true]{hyperref}
\usepackage[nameinlink]{cleveref}
\usepackage{footnotebackref}
\usepackage{url}
\usepackage{listings}
\usepackage[most]{tcolorbox}
\usepackage{inconsolata}
\usepackage[papersize={8.5in,11in}, margin=1in]{geometry}
\usepackage{float}
\usepackage{caption}
\usepackage{esint}
\usepackage{url}
\usepackage{enumitem}
\usepackage{subfig}
\usepackage{wasysym}
\newcommand{\ilc}{\texttt}
\usepackage{etoolbox}
\usepackage{algorithm}
\usepackage{changepage}
% \usepackage{algorithmic}
\usepackage[noend]{algpseudocode}
\usepackage{tikz}
\usetikzlibrary{matrix,positioning,arrows.meta,arrows}
\patchcmd{\thebibliography}{\section*{\refname}}{}{}{}
% \PassOptionsToPackage{hyphens}{url}\usepackage{hyperref}

\providecommand{\myceil}[1]{\left \lceil #1 \right \rceil }
\providecommand{\myfloor}[1]{\left \lfloor #1 \right \rfloor }


\begin{document}



\title{\textbf{MATH 307: Group Homework 2}}


\author{\textit{Group 8}\\
Shaochen (Henry) ZHONG, Zhitao (Robert) CHEN, John MAYS, Huaijin XIN\\ \ilc{\{sxz517, zxc325, jkm100, hxx200\}@case.edu}}

\date{Due and submitted on 02/12/2021 \\ Spring 2021, Dr. Guo}
\maketitle



\section*{Problem 1}

\textit{ Prove that $\mathbb{C}$, the set of all complex valued numbers with the regular addition and multiplication of complex numbers is a field.}\newline


%Some wording taken verbatim from Henry's IHW2:
For the simplicity of discussion, say we have three complex numbers: $z_1,z_2,z_3 \in \mathbb{C}$ that can be represented respectively as:
\begin{equation*}
\begin{split}
    z_1&=a+ib\\
    z_2&=c+id\\
    z_3&=e+if\\
\end{split}
\end{equation*}

where $i^2 = -1$ and $a,b,c,d,e,f \in \mathbb{R}$.\newline

\noindent $(\mathbb{C}, +, \times)$ is a field, because it satisfies all of the conditions for a field:
\subsubsection*{\textbf{1. }$(\mathbb{C}, +)$ is an Abelian group:}
\begin{enumerate}
    \item Closure: $\forall z_1,z_2 \in \mathbb{C}, z_1+z_2 =(a+ib)+(c+id) = (a+b)+i(c+d) \in \mathbb{C}$
    \item Associativity: $\forall z_1,z_2,z_3 \in \mathbb{C},z_1+(z_2+z_3)=(z_1+z_2)+z_3$\\
    Proof:\\
    $z_1+(z_2+z_3)=a+ib+((c+e)+i(d+f))=(a+c+e)+i(b+d+f)\\(z_1+z_2)+z_3=((a+c)+i(b+d))+e+if=(a+c+e)+i(b+d+f)$
    \item Identity: $\forall z \in \mathbb{C}, \exists id=0+i0$ s.t $z+id=id+z=z$\\
    Proof:\\
    $z+(0+i0)=(a+0)+i(b+0)=a+ib=z$\\
    $(0+i0)+z=(0+a)+i(0+b)=a+ib=z$
    \item Inverse: $\forall z \in \mathbb{C}, \exists -z$ s.t $z+(-z)=0+i0$\\
    Proof:\\
    $z+(-z)=a+ib-(a+ib)=(a-a)+i(b-b)=0+0i$
    \item Commutativity: $\forall z_1,z_2 \in \mathbb{C}, z_1+z_2=z_2+z_1$\\
    Proof:\\
    $z_1+z_2=a+ib+c+id=(a+c)+i(b+d)$\\
    $z_2+z_1=c+id+a+ib=(a+c)+i(d+b)$\\
\end{enumerate}
​
\subsubsection*{\textbf{2. }$\mathbb{C}, \times$ is a commutative monoid:}
\begin{enumerate}
    \item Closure: $\forall z_1,z_2 \in \mathbb{C}, z_1\times z_2 =(a+ib)\times (c+id) = (ac-bd)+i(bc+ad) \in \mathbb{C}$
    \item Associativity: $\forall z_1,z_2,z_3 \in \mathbb{C},z_1\times (z_2\times z_3)=(z_1\times z_2)\times z_3$\\
    Proof:\\
    $z_1\times(z_2\times z_3)=(a+ib)\times((c+id)\times(e+if)=(a+ib)\times((ce-df)+i(de+cf))=(ace-adf-bde-bcf)+i(bce-dfb+ade+acf)$\\
    $(z_1\times z_2)\times z_3=((a+ib)\times(c+id))\times(e+if)=((ac-bd)+i(bc+da))\times(e+if)=(eac-ebd-fbc-fda)+i(bce+dae+fac-fbd)$
    \item Identity: $\forall z \in \mathbb{C}, \exists id=1+i0$ s.t $z\times id=id\times z=z$\\
    Proof:\\
    $z\times id=(a+ib)\times (1+i0)=(a-0)+i(b+0)=a+ib=z$\\
    $id\times z=(1+i0)\times (a+ib)=(a-0)+i(b+0)=a+ib=z$
    \item Commutativity: $\forall z_1,z_2 \in \mathbb{C}, z_1\times z_2=z_2\times z_1$\\
    Proof:\\
    $z_1\times z_2=(a+ib)\times(c+id)=(ac-bd)+i(cb+da)$\\
    $z_2\times z_1=(c+id)\times(a+ib)=(ac-bd)+i(cb+da)$
\end{enumerate}
​
\subsubsection*{\textbf{3. }$\forall z \in \mathbb{C}\backslash\{0+i0\}, \exists z^{-1}$ s.t. $z\times z^{-1}=1$}
Proof:\\
Assume $z=a+ib:$\\
$z^{-1}=\frac{1}{z}=\frac{1}{a+ib}=\frac{a-ib}{(a+ib)\times(a-ib)}=\frac{a-ib}{a^2+b^2}$\\
Now, $z\times z^{-1}=a+ib \times \frac{a-ib}{a^2+b^2} =\frac{(a-ib)\times (a+ib)}{a^2+b^2}=\frac{a^2+b^2}{a^2+b^2}=1$
​
\subsubsection*{\textbf{4. }$\times$ is distributive over $+$ s.t. $\forall z_1,z_2,z_3 \in \mathbb{C},$ it holds that $z_1\times(z_2+z_3)=z_1\times z_2 + z_1 \times z_3$}
Proof:\\
$z_1\times(z_2+z_3)=(a+ib)\times((c+id)+(e+if))=(a+ib)\times((c+e)+i(d+f))=(ac+ae-bd-bf)+i(bc+be+ad+af)=(ac-bd)+i(bc+ad)+(ae-bf)+i(be+fa)=(a+ib)\times(c+id)+(a+ib)\times(e+if)=z_1\times z_2 + z_1 \times z_3$



\section*{Problem 2}

\textit{Compute the multiplicative inverse of $z = 3 - 7i$ and express it in the form $a+ib$.}\newline

Known that for $a + bi \neq 0$, we have its multiplicative inverse being $\frac{a}{a^2 + b^2} - i\frac{b}{a^2 + b^2} $.
\begin{align*}
    (3 - 7i)^{-1} &= \frac{3}{3^2 + 7^2} - i\frac{-7}{3^2 + 7^2} \\
    &= \frac{3}{58} + i\frac{7}{58}
\end{align*}

\section*{Problem 3}
\textit{Write the complex number $z = -1 - i$ in polar form, then compute $z^4$.}\newline

We have the polar form of $z = -1 - i$ being:

\begin{align*}
    z &= \sqrt{1^2 + 1^2}(\frac{-1}{1^2 + 1^2} - i\frac{-1}{1^2 + 1^2}) \\
    &= \sqrt{2}(\cos (\pi + \tan^{-1}(\frac{-1}{-1})) + i\sin( \pi + \tan^{-1}(\frac{-1}{-1})) ) \\
    &= \sqrt{2}(\cos (\frac{5 \pi}{4}) + i\sin (\frac{5 \pi}{4}) )
\end{align*}

Now to calculate $z^4$ with \textsc{De Moivre's} formula:

\begin{align*}
    z^4 &= (\sqrt{2})^4(\cos (4 \cdot \frac{5 \pi}{4}) + i\sin (4 \cdot \frac{5 \pi}{4}) ) \\
    &= 4 (\cos 5\pi + i\sin 5\pi ) \\
    &= 4(-1 + 0) = -4
\end{align*}

\section*{Problem 4}
\textit{Using the polar form, show that the product of a complex number $z = a + ib$ and its complex conjugate must be a real number greater than or equal to zero.}\newline

For a complex number $z = a + ib$, let $r =\sqrt{a^2 + b^2}$ and $\theta = \tan^{-1}(\frac{b}{a})$, we may express its as polar form as
$r(\cos \theta + i\sin \theta)$.

Then we have:

\begin{align*}
    \overline{z} &= \overline{r} \cdot \overline{(\cos \theta + i\sin \theta)} \\
    &= r \cdot  \overline{(\cos \theta + i\sin \theta)}  \\
    &= r (\cos \theta - i\sin \theta)
\end{align*}

Then for $z \cdot \overline{z}$, we have:

\begin{align*}
    z \cdot \overline{z} &= r (\cos \theta + i\sin \theta) \cdot r (\cos \theta - i\sin \theta) \\
    &= (r \cos \theta  + r i\sin \theta) \cdot (r \cos \theta  - r i\sin \theta) \\
    &= r^2 (\cos \theta)^2 - r^2 \cos \theta i \sin \theta + r^2 i \sin \theta \cos \theta - r^2 (i \sin \theta)^2 \\
    &= r^2 (\cos \theta)^2 +  r^2 (\sin \theta)^2 \\
    &= r^2
\end{align*}

As there is no imaginary part in the resultant equation, and as $r^2 = (a^2 + b^2)$ is always $\geq 0$, we may conclude that $z \cdot \overline{z}$ must be a real number greater than or equal to zero

\end{document}


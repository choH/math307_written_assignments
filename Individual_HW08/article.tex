\documentclass[11pt]{article}
\usepackage{setspace}
\setstretch{1}
\usepackage{amsmath,amssymb, amsthm}
\usepackage{graphicx}
\usepackage{bm}
\usepackage[hang, flushmargin]{footmisc}
\usepackage[colorlinks=true]{hyperref}
\usepackage[nameinlink]{cleveref}
\usepackage{footnotebackref}
\usepackage{url}
\usepackage{listings}
\usepackage[most]{tcolorbox}
\usepackage{inconsolata}
\usepackage[papersize={8.5in,11in}, margin=1in]{geometry}
\usepackage{float}
\usepackage{caption}
\usepackage{esint}
\usepackage{url}
\usepackage{enumitem}
\usepackage{subfig}
\usepackage{wasysym}
\newcommand{\ilc}{\texttt}
\usepackage{etoolbox}
\usepackage{algorithm}
\usepackage{changepage}
% \usepackage{algorithmic}
\usepackage[noend]{algpseudocode}
\usepackage{tikz}
\usetikzlibrary{matrix,positioning,arrows.meta,arrows}
\patchcmd{\thebibliography}{\section*{\refname}}{}{}{}
% \PassOptionsToPackage{hyphens}{url}\usepackage{hyperref}

\providecommand{\myceil}[1]{\left \lceil #1 \right \rceil }
\providecommand{\myfloor}[1]{\left \lfloor #1 \right \rfloor }
\providecommand{\qbm}[1]{\begin{bmatrix} #1 \end{bmatrix}}
\providecommand{\qpm}[1]{\begin{pmatrix} #1 \end{pmatrix}}
\providecommand{\norm}[1]{\left\lVert #1 \right\rVert}
\providecommand{\len}[1]{\left| #1 \right|}

\begin{document}



\title{\textbf{MATH 307: Individual Homework 8}}


\author{Shaochen (Henry) ZHONG, \ilc{sxz517@case.edu}}

\date{Due and submitted on 03/03/2021 \\ Spring 2021, Dr. Guo}
\maketitle



\subsection*{Problem 1}
\textit{See HW instruction.}\newline

For the ease of expression let the following $\norm{x}$ be $\norm{x}_1$ in this problem.

We know $\sum_i^n \len{x_i}$ is a norm as it shows \textit{non-negativity} as we always have $\sum_i^n \len{x_i} \geq 0$ as $\len{x_i} = \sqrt{Re(x_i)^2 + Im(x_i)^2} \geq 0$.

It also shows \textit{scaling} as for $\norm{\lambda x}$ for all $\lambda \in \mathbb{R}$ we have:

\begin{align*}
    \norm{\lambda x} &= \sum_i^n \len{\lambda x_i} \\
    &= \sum_i^n \len{\lambda (\sqrt{Re(x_i)^2 + IM(x_i)^2})} \\
    &= \lambda \sum_i^n \len{x_i} = \len{\lambda} \norm{x}
\end{align*}

For \textit{Triangle inequality}, assume we have a $y = [y_1, y_2, y_3, \dots, y_n]$ for $y_i \in \mathbb{C}$. We must have:

\begin{align*}
    \norm{x + y} &= \norm{\sum_i^n \len{x_i + y_i}} =\sum_i^n \len{x_i + y_i} \\
    \sum_i^n \len{x_i + y_i} &\leq \sum_i^n \len{x_i} + \len{y_i} = \norm{x} + \norm{y} \\
    \norm{x + y} &\leq \norm{x} + \norm{y}
\end{align*}

As all conditions for norm are proven, we may confirm $\sum_i^n \len{x_i}$ is a norm.

\subsection*{Problem 2}
\textit{See HW instruction.}\newline

For the ease of expression we let $\len{x_k}$ to have the maximum value among all $\len{x_i}$. We first to show $\norm{x}_{\infty} \leq \norm{x}_1$:

\begin{align*}
    \norm{x}_{\infty} &= \len{x_k} \\
    \norm{x}_1 &= \sum_i^n \len{x_i} = \len{x_1} + \len{x_2} + \dots + \len{x_k} + \dots + \len{x_n} \\
    \text{Since all} &\ \len{x_i} \geq 0 \\
    \Longrightarrow \norm{x}_{\infty} &\leq \norm{x}_1
\end{align*}

Now to show that $\norm{x}_1 \leq n\norm{x}_{\infty}$:

\begin{align*}
    \norm{x}_1 &= \sum_i^n \len{x_i} = \underbrace{\len{x_1} + \len{x_2} + \dots + \len{x_k} + \dots + \len{x_n}}_{\text{$n$ elements}} \\
    n\norm{x}_{\infty} &= n \cdot \len{x_k} \\
    \text{Since} &\ \len{x_k} \geq \len{x_i} \ \text{for all} \ \len{x_i}\\
    \Longrightarrow \norm{x}_1 &\leq n\norm{x}_{\infty}
\end{align*}

By combining the above two findings together we have $\norm{x}_{\infty} \leq \norm{x}_1  \leq n\norm{x}_{\infty}$. Thus, the $1$ and $\infty$ norm are equivalent in $\mathbb{C}^n$.

\end{document}


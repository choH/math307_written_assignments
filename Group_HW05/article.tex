\documentclass[11pt]{article}
\usepackage{setspace}
\setstretch{1}
\usepackage{amsmath,amssymb, amsthm}
\usepackage{graphicx}
\usepackage{bm}
\usepackage[hang, flushmargin]{footmisc}
\usepackage[colorlinks=true]{hyperref}
\usepackage[nameinlink]{cleveref}
\usepackage{footnotebackref}
\usepackage{url}
\usepackage{listings}
\usepackage[most]{tcolorbox}
\usepackage{inconsolata}
\usepackage[papersize={8.5in,11in}, margin=1in]{geometry}
\usepackage{float}
\usepackage{caption}
\usepackage{esint}
\usepackage{url}
\usepackage{enumitem}
\usepackage{subfig}
\usepackage{wasysym}
\newcommand{\ilc}{\texttt}
\usepackage{etoolbox}
\usepackage{algorithm}
\usepackage{changepage}
% \usepackage{algorithmic}
\usepackage[noend]{algpseudocode}
\usepackage{tikz}
\usepackage{gensymb}
\usetikzlibrary{matrix,positioning,arrows.meta,arrows}
\patchcmd{\thebibliography}{\section*{\refname}}{}{}{}
% \PassOptionsToPackage{hyphens}{url}\usepackage{hyperref}

\providecommand{\myceil}[1]{\left \lceil #1 \right \rceil }
\providecommand{\myfloor}[1]{\left \lfloor #1 \right \rfloor }
\providecommand{\qbm}[1]{\begin{bmatrix} #1 \end{bmatrix}}
\providecommand{\qpm}[1]{\begin{pmatrix} #1 \end{pmatrix}}
\providecommand{\norm}[1]{\left\lVert #1 \right\rVert}
\providecommand{\len}[1]{\left| #1 \right|}

\begin{document}




\title{\textbf{MATH 307: Group Homework 5}}


\author{\textit{Group 8}\\
Shaochen (Henry) ZHONG, Zhitao (Robert) CHEN, John MAYS, Huaijin XIN\\ \ilc{\{sxz517, zxc325, jkm100, hxx200\}@case.edu}}

\date{Due and submitted on 03/12/2021 \\ Spring 2021, Dr. Guo}
\maketitle

% P_{v}(w) = < w, v/|v| > v/|v| project w onto v


\subsection*{Problem 1}
\textit{See HW instruction.}\newline

The vector-in-question can be transform as:

\begin{align*}
    &= x_1 \qbm{1 \\ 1 \\ 0 \\ 1} + x_2 \qbm{-1 \\ 0 \\ 0 \\ 3} + x_3 \qbm{2 \\ 2 \\ 3 \\0} + x_4 \qbm{0 \\ 0 \\ 0 \\  3} \\
    &= \qbm{1 & -1 & 2 & 0 \\
            1 & 0 & 2 & 0 \\
            0 & 0 & 3 & 0 \\
            1 & 3 & 0 & 3} \cdot \qbm{x_1 \\ x_2 \\ x_3 \\ x_4 }
\end{align*}

\subsection*{Problem 2}
\textit{See HW instruction.}\newline

Assume the original vector $\qbm{x_0 \\ y_0}$ with a length of $r$ has a degree of $\theta$, we first reflect it about y-axis by swapping the $x$ and $y$ value and make $A_1 \qbm{x_0 \\ y_0} = \qbm{x_0 \\ -y_0}$, this means $A_1 = \qbm{-1 & 0 \\ 0 & 1}$.


Say the $\qbm{x_0 \\ -y_0}$, we call it $\qbm{x_1 \\ y_1}$, got a degree of $\theta$, we then rotate it $\phi$ (in this case $\phi = 90\degree$) degrees more couterclock wisely to have $\qbm{x_2 \\ y_2}$:

\begin{align*}
    x_2 &= r \cos(\theta + \phi) = r(\cos \theta \cos \phi - \sin \theta \sin \phi) \\
    &= r \cos \theta \cos \phi - r \sin \theta \sin \phi \\
    &= x_1 \cos \phi - y_1 \sin \phi \\
    y_2 &= r \sin(\theta + \phi) = r(\sin \theta \cos \phi + \cos \theta \sin \phi) \\
    &= r\sin \theta \cos \phi + r\cos \theta \sin \phi \\
    &= y_1 \cos \phi + x_1 \sin \phi
\end{align*}

As we need $A \qbm{x_1 \\ y_1} = \qbm{x_2 \\ y_2}$, we must have:

\begin{align*}
    x_2 &= x_1 \cos \phi - y_1 \sin \phi \\
    y_2 &= y_1 \cos \phi + x_1 \sin \phi \\
    \qbm{x_2 \\ y_2} &= \qbm{\cos \phi & -\sin \phi \\ \sin \phi & \cos \phi} \cdot \qbm{x_1 \\ y_1} \\
    &\text{Since}\ \phi = 90\degree \\
    \qbm{x_2 \\ y_2} &= \qbm{0 & -1 \\ 1 & 0} \cdot \qbm{x_1 \\ y_1} \\
    &= \qbm{0 & -1 \\ 1 & 0} \cdot (\qbm{-1 & 0 \\ 0 & 1} \qbm{x_0 \\ y_0}) \\
    &= (\qbm{0 & -1 \\ 1 & 0} \cdot \qbm{-1 & 0 \\ 0 & 1}) \qbm{x_0 \\ y_0} \\
    \Longrightarrow A &= \qbm{0 & -1 \\ -1 & 0}
\end{align*}

Thus, $A = \qbm{0 & -1 \\ -1 & 0}$.

\subsection*{Problem 3}
\textit{See HW instruction.}\newline

$A$ is not a square matrix but a \textbf{retangular} one as it has unequal number of rows and columns. So it is also not a diagonal, upper-/lower-triangular as these require being a square matrix as prerequisite.

$B$ is a \textbf{square} matrix (and therefore not retangular) as it has equal number of rows and columns. It is not diagonal, as we has non-zero entries outside of the main diagonal. It is a \textbf{lower-triangular} matrix as all entries above the main diagonal are zero.

$C$ is a \textbf{square} matrix (and therefore not retangular) as it has equal number of rows and columns. It is also \textbf{diagonal} as it has zero entries except the main diagonal; it is also both an \textbf{upper-} and a \textbf{lower-triangular} matrix as all entries below/above its main diagonal are zero.

\end{document}


\documentclass[11pt]{article}
\usepackage{setspace}
\setstretch{1}
\usepackage{amsmath,amssymb, amsthm}
\usepackage{graphicx}
\usepackage{bm}
\usepackage[hang, flushmargin]{footmisc}
\usepackage[colorlinks=true]{hyperref}
\usepackage[nameinlink]{cleveref}
\usepackage{footnotebackref}
\usepackage{url}
\usepackage{listings}
\usepackage[most]{tcolorbox}
\usepackage{inconsolata}
\usepackage[papersize={8.5in,11in}, margin=1in]{geometry}
\usepackage{float}
\usepackage{caption}
\usepackage{esint}
\usepackage{url}
\usepackage{enumitem}
\usepackage{subfig}
\usepackage{wasysym}
\newcommand{\ilc}{\texttt}
\usepackage{etoolbox}
\usepackage{algorithm}
\usepackage{changepage}
% \usepackage{algorithmic}
\usepackage[noend]{algpseudocode}
\usepackage{tikz}
\usetikzlibrary{matrix,positioning,arrows.meta,arrows}
\patchcmd{\thebibliography}{\section*{\refname}}{}{}{}
% \PassOptionsToPackage{hyphens}{url}\usepackage{hyperref}

\providecommand{\myceil}[1]{\left \lceil #1 \right \rceil }
\providecommand{\myfloor}[1]{\left \lfloor #1 \right \rfloor }
\providecommand{\qbm}[1]{\begin{bmatrix} #1 \end{bmatrix}}
\providecommand{\qpm}[1]{\begin{pmatrix} #1 \end{pmatrix}}


\begin{document}


\title{\textbf{MATH 307: Group Homework 3}}


\author{\textit{Group 8}\\
Shaochen (Henry) ZHONG, Zhitao (Robert) CHEN, John MAYS, Huaijin XIN\\ \ilc{\{sxz517, zxc325, jkm100, hxx200\}@case.edu}}

\date{Due and submitted on 02/19/2021 \\ Spring 2021, Dr. Guo}
\maketitle



\section*{Problem 1}

For $u = \begin{pmatrix} 1 \\ 2\end{pmatrix}$, $v = \begin{pmatrix} 1 \\ 1\end{pmatrix}$, and $w = \begin{pmatrix} -1 \\ 0\end{pmatrix}$. We may obtain $ \begin{pmatrix} 1 \\ 0 \end{pmatrix} = -w$ and $\begin{pmatrix} 0 \\ 1\end{pmatrix} = v + w$. So for any arbitrary $\begin{pmatrix} x \\ y\end{pmatrix} \in \mathbb{R}^2$ we have:

\begin{align*}
    \begin{pmatrix} x \\ y\end{pmatrix} &= y u + (y-x) w \\
    &= y \begin{pmatrix} 1 \\ 1\end{pmatrix} + (y-x)\begin{pmatrix} -1 \\ 0\end{pmatrix}
\end{align*}

Thus, we have $span(u, v, w) \subset \mathbb{R}$.

For any arbitrary vector $d \in span(u, v, w)$, we have:

\begin{align*}
    d &= \lambda_1 u + \lambda_2 v + \lambda_3 w \\
    &= \begin{pmatrix}
        \lambda_1 + \lambda_2 - \lambda_3 \\
        2\lambda_1 + \lambda_2\\
    \end{pmatrix}
\end{align*}

where $d \in \mathbb{R}^2$, thus we have $\mathbb{R} \subset span(u, v, w)$ and therefore $\mathbb{R} = span(u, v, w)$

\end{document}


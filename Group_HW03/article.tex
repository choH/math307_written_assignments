\documentclass[11pt]{article}
\usepackage{setspace}
\setstretch{1}
\usepackage{amsmath,amssymb, amsthm}
\usepackage{graphicx}
\usepackage{bm}
\usepackage[hang, flushmargin]{footmisc}
\usepackage[colorlinks=true]{hyperref}
\usepackage[nameinlink]{cleveref}
\usepackage{footnotebackref}
\usepackage{url}
\usepackage{listings}
\usepackage[most]{tcolorbox}
\usepackage{inconsolata}
\usepackage[papersize={8.5in,11in}, margin=1in]{geometry}
\usepackage{float}
\usepackage{caption}
\usepackage{esint}
\usepackage{url}
\usepackage{enumitem}
\usepackage{subfig}
\usepackage{wasysym}
\newcommand{\ilc}{\texttt}
\usepackage{etoolbox}
\usepackage{algorithm}
\usepackage{changepage}
% \usepackage{algorithmic}
\usepackage[noend]{algpseudocode}
\usepackage{tikz}
\usetikzlibrary{matrix,positioning,arrows.meta,arrows}
\patchcmd{\thebibliography}{\section*{\refname}}{}{}{}
% \PassOptionsToPackage{hyphens}{url}\usepackage{hyperref}

\providecommand{\myceil}[1]{\left \lceil #1 \right \rceil }
\providecommand{\myfloor}[1]{\left \lfloor #1 \right \rfloor }
\providecommand{\qbm}[1]{\begin{bmatrix} #1 \end{bmatrix}}
\providecommand{\qpm}[1]{\begin{pmatrix} #1 \end{pmatrix}}


\begin{document}


\title{\textbf{MATH 307: Group Homework 3}}


\author{\textit{Group 8}\\
Shaochen (Henry) ZHONG, Zhitao (Robert) CHEN, John MAYS, Huaijin XIN\\ \ilc{\{sxz517, zxc325, jkm100, hxx200\}@case.edu}}

\date{Due and submitted on 02/19/2021 \\ Spring 2021, Dr. Guo}
\maketitle



\section*{Problem 1}

For $u = \begin{pmatrix} 1 \\ 2\end{pmatrix}$, $v = \begin{pmatrix} 1 \\ 1\end{pmatrix}$, and $w = \begin{pmatrix} -1 \\ 0\end{pmatrix}$. We may obtain $ \begin{pmatrix} 1 \\ 0 \end{pmatrix} = -w$ and $\begin{pmatrix} 0 \\ 1\end{pmatrix} = v + w$. So for any arbitrary $\begin{pmatrix} x \\ y\end{pmatrix} \in \mathbb{R}^2$ we have:

\begin{align*}
    \begin{pmatrix} x \\ y\end{pmatrix} &= y u + (y-x) w \\
    &= y \begin{pmatrix} 1 \\ 1\end{pmatrix} + (y-x)\begin{pmatrix} -1 \\ 0\end{pmatrix}
\end{align*}

Thus, we have $span(u, v, w) \subset \mathbb{R}^2$.

For any arbitrary vector $d \in span(u, v, w)$, we have:

\begin{align*}
    d &= \lambda_1 u + \lambda_2 v + \lambda_3 w \\
    &= \begin{pmatrix}
        \lambda_1 + \lambda_2 - \lambda_3 \\
        2\lambda_1 + \lambda_2\\
    \end{pmatrix}
\end{align*}

where $d \in \mathbb{R}^2$, thus we have $\mathbb{R}^2 \subset span(u, v, w)$ and therefore $\mathbb{R}^2 = span(u, v, w)$

\section*{Problem 2}

$\forall p \in P^4 = a_0 + a_1 x + a_2 x^2 + a_3 x^3 + a_4 x^4$ for $a \in \mathbb{R}$, we will have a $span(-x^4, x^3, -x^2, x, -1) \subset P^4$.

\begin{equation*}
    -a_0 \cdot (-1) + a_1 x - a_2 (-x^2) + a_3 x^3 - a_4 (-x^4)
\end{equation*}

which is equals to such $p$, which implies $span(-x^4, x^3, -x^2, x, -1) \subset P^4$.\newline

\noindent $\forall q \in span(-x^4, x^3, -x^2, x, -1)$, we may express them as $q = b_4(-x^4) + b_3(x^3) + b_2(-x^2) + b_1(x) + (b_0)(- x^0)$ which is clearly in $\in P^4$, so we also have $P^4 \subset span(-x^4, x^3, -x^2, x, -1)$ and therefore $P^4 = span(-x^4, x^3, -x^2, x, -1)$.

\section*{Problem 3}

\subsection*{(a)}

No, it is linearly dependent as we may have:

% \begin{align*}
%     0 &= 1 \begin{bmatrix} 1 \\ 1 + i\end{bmatrix} + (-\frac{1}{2} - \frac{1}{2}i) \begin{bmatrix} 1-i \\ 2\end{bmatrix}
% \end{align*}

\begin{align*}
    0 &= \lambda_1 \qbm{1 \\ 1+i} + \lambda_2 \qbm{1-i \\ 2} \\
    &\begin{cases}
        \lambda_1 + \lambda_2 - \lambda_2 i = 0 \\
        \lambda_1 + \lambda_1 i - 2\lambda_2 = 0
    \end{cases} \\
    \Longrightarrow& \begin{cases}
            \lambda_1 = -3 + i \\
            \lambda_2 = 2 + i
        \end{cases}
\end{align*}

for $\lambda_1, \lambda_2 \in \mathbb{C}$, where the coefficients are not zeros.

\subsection*{(b)}

No, it is linearly dependent as we may have:


\begin{align*}
    0 &= \lambda_1 \qbm{1\\1\\2} + \lambda_2 \qbm{2 \\1 \\3} + \lambda_3 \qbm{1 \\ 0 \\1} \\
    \Longrightarrow& \lambda_1 = -\lambda_2 = \lambda_3
\end{align*}


for $\lambda_1, \lambda_2, \lambda_3 \in \mathbb{C}$, where the coefficients are not zeros.

\subsection*{(c)}

No, it is linearly dependent as we may have:


\begin{align*}
    0 &= \lambda_1 \qbm{ -1 \\ 2 \\ 0} + \lambda_2 \qbm{ 2 \\ -3 \\ 1 } + \lambda_3 \qbm{ 0 \\ 4 \\ 5 } + \lambda_4 \qbm{ 1 \\ -2 \\ 1 }\\
    \Longrightarrow& \begin{cases}
            \lambda_1 &= -7 \lamabda_4 \\
            \lambda_2 &= -4 \lambda_4  \\
            \lambda_3 &= \lambda_4
    \end{cases}
\end{align*}

Assume $\lambda_4 = 1$, we have $\lambda_3 = 4, \lambda_2 = -4, \lambda_1 = -7$ which is a non zero solution of the system.


\section*{Problem 4}

For a vector space of $\mathbb{C}^{2 \times 3}$, we may have an arbitrary matrix like $\qbm{a + bi & e + fi & j + ki \\ c + di & g + hi & l + mi}$ for $a, b, ..., h, j, ..., m \in \mathbb{R}$. So by having a the following system:

\begin{equation*}
    \underbrace{ \qbm{1 & 0 & 0 \\ 0 & 0 & 0 }, \qbm{0 & 0 & 0 \\ 1 & 0 & 0 }, \cdots, \qbm{0 & 0 & 0 \\ 0 & 0 & 1}}_\text{6 matrices}
\end{equation*}

We first know they are linearly independent as to have $\lambda_1 \qbm{1 & 0 & 0 \\ 0 & 0 & 0 } +\lambda_2  \qbm{0 & 0 & 0 \\ 1 & 0 & 0 } +  \cdots + \lambda_6 \qbm{0 & 0 & 0 \\ 0 & 0 & 1} = 0$ we must have $\lambda_1 = \lambda_2 = ... = \lambda_6 = 0$.

Then for spanning, by simply assigning the scalar mutipliers of $(a + bi), (c + di), \cdots, (l + mi)$ to these matrices respectively, we shall produce any $\qbm{a + bi & e + fi & j + ki \\ c + di & g + hi & l + mi} \in \mathbb{C}^{2 \times 3}$. Thus, the proposed system is a basis of $\mathbb{C}^{2 \times 3}$.

\section*{Problem 5}

Yes, first we know they are linearly independent as we may only have:

\begin{align*}
    0 &= \lambda_1 \begin{bmatrix} 1 \\ i \end{bmatrix} + \lambda_2 \begin{bmatrix} 0 \\ 1 \end{bmatrix} \\
    &= 0 \begin{bmatrix} 1 \\ i \end{bmatrix} + 0 \begin{bmatrix} 0 \\ 1 \end{bmatrix}
\end{align*}
where the coefficients $\lambda_1 = \lambda_2 = 0$.

Then to prove they are spanning over $\mathbb{C}^2$, we may have any arbitrary $\begin{bmatrix} a + bi \\ c + di \end{bmatrix}$ for $a, b, c, d \in \mathbb{R}$. This means if we may individually produce $(1, 0), (0, 1), (i, 0), (0, i)$, we shall simply put $a, c, b, d$ as their scalar mutipliers respectively then add them together, we have any possible $\begin{bmatrix} a + bi \\ c + di \end{bmatrix}$.

\begin{align*}
    \qbm{0 \\ 1} &= \qbm{0 \\ 1} \\
    \qbm{1 \\ 0} &= \qbm{1 \\ i} - i\qbm{0 \\ 1} \\
    \qbm{i \\ 0} &= i\qbm{1 \\ i} + \qbm{0 \\ 1} = \qbm{i \\ -1} + \qbm{0 \\ 1} \\
    \qbm{0 \\ i} &=  i\qbm{0 \\ 1}
\end{align*}

Thus $\qbm{1 \\ i}, \qbm{0 \\ 1}$ is also spanning $\mathbb{C}^2$ and therefore a basis of it.

\end{document}


\documentclass[11pt]{article}
\usepackage{setspace}
\setstretch{1}
\usepackage{amsmath,amssymb, amsthm}
\usepackage{graphicx}
\usepackage{bm}
\usepackage[hang, flushmargin]{footmisc}
\usepackage[colorlinks=true]{hyperref}
\usepackage[nameinlink]{cleveref}
\usepackage{footnotebackref}
\usepackage{url}
\usepackage{listings}
\usepackage[most]{tcolorbox}
\usepackage{inconsolata}
\usepackage[papersize={8.5in,11in}, margin=1in]{geometry}
\usepackage{float}
\usepackage{caption}
\usepackage{esint}
\usepackage{url}
\usepackage{enumitem}
\usepackage{subfig}
\usepackage{wasysym}
\newcommand{\ilc}{\texttt}
\usepackage{etoolbox}
\usepackage{algorithm}
\usepackage{changepage}
% \usepackage{algorithmic}
\usepackage[noend]{algpseudocode}
\usepackage{tikz}
\usetikzlibrary{matrix,positioning,arrows.meta,arrows}
\patchcmd{\thebibliography}{\section*{\refname}}{}{}{}
% \PassOptionsToPackage{hyphens}{url}\usepackage{hyperref}

\providecommand{\myceil}[1]{\left \lceil #1 \right \rceil }
\providecommand{\myfloor}[1]{\left \lfloor #1 \right \rfloor }
\providecommand{\qbm}[1]{\begin{bmatrix} #1 \end{bmatrix}}
\providecommand{\qpm}[1]{\begin{pmatrix} #1 \end{pmatrix}}
\providecommand{\norm}[1]{\left\lVert #1 \right\rVert}

\begin{document}


\title{\textbf{MATH 307: Group Homework 4}}


\author{\textit{Group 8}\\
Shaochen (Henry) ZHONG, Zhitao (Robert) CHEN, John MAYS, Huaijin XIN\\ \ilc{\{sxz517, zxc325, jkm100, hxx200\}@case.edu}}

\date{Due and submitted on 02/26/2021 \\ Spring 2021, Dr. Guo}
\maketitle



\section*{Problem 1}
\textit{Textbook page 42, problem 22.}\newline

Since we know that $\norm{x} = \sqrt{<x, x>}$; and due to conjugate symmetry we know that $<x, y> = \overline{y, x}$, implying $<x, y> + <y, x> =  2 Re<x, y>$. Thus, for $x, y \in V$, we may have:

\begin{align*}
    \norm{x + y}^2 &= <x + y, x + y> \\
    &= <x, x> + <x, y> + <y, x> + <y, y> \\
    &= \norm{x}^2 + 2 Re<x, y> + \norm{y}^2 \\
    \norm{x - y}^2 &= <x - y, x - y> \\
    &= <x, x> + (-<x, y>) + (-<y, x>) + <y, y> \\
    &= \norm{x}^2 - 2 Re<x, y>+ \norm{y}^2 \\
    \Longrightarrow& \norm{x + y}^2 + \norm{x - y}^2 = 2\norm{x}^2 + 2\norm{y}^2
\end{align*}

The equality-in-question is therefore shown.

\section*{Problem 2}
\textit{See HW instruction.}\newline

Known from the above problem, we have the following (note we don't have to specify $Re<x, y>$ here as the problem specified $V$ is a vector space with a real valued inner product, so $Re<x, y> = <x, y>$):

\begin{align*}
    \norm{x + y}^2 - \norm{x - y}^2 &= \norm{x}^2 + 2<x, y> + \norm{y}^2  - (\norm{x}^2 - 2<x, y> + \norm{y}^2) \\
    &= 4<x, y> \\
    \Longrightarrow <x, y> &= \frac{1}{4}(\norm{x + y}^2 + \norm{x - y}^2)
\end{align*}

The equality-in-question is therefore shown.

\section*{Problem 3}
\textit{See HW instruction.}\newline

Known from the above two problem, we have:
\begin{align*}
    \norm{u + v}^2 &= <u + v, u + v> \\
    &= \norm{u}^2 + 2 Re<u, v> + \norm{v}^2 = 7^2 \\
    \norm{u - v}^2 &= <u - v, u - v> \\
    &= \norm{u}^2 - 2 Re<u, v> + \norm{v}^2 = 3^2 \\
    \Rightarrow <u + v, u + v> &- <u - v, u - v> = 4 Re <u, v> = 49 - 9 = 40 \\
    \Longrightarrow& \ Re<u, v> = 10 \\
    \Rightarrow <u + v, u + v> &+ <u - v, u - v> = 2\norm{u}^2 + 2\norm{v}^2 = 49 + 9 = 58 \\
    \Longrightarrow& \ \norm{u}^2 + \norm{v}^2 = 29 \\
\end{align*}

Thus, we have $<u, v> = 10$ and $\norm{u}^2 + \norm{v}^2 = 29$.

\end{document}


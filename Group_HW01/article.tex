\documentclass[11pt]{article}
\usepackage{setspace}
\setstretch{1}
\usepackage{amsmath,amssymb, amsthm}
\usepackage{graphicx}
\usepackage{bm}
\usepackage[hang, flushmargin]{footmisc}
\usepackage[colorlinks=true]{hyperref}
\usepackage[nameinlink]{cleveref}
\usepackage{footnotebackref}
\usepackage{url}
\usepackage{listings}
\usepackage[most]{tcolorbox}
\usepackage{inconsolata}
\usepackage[papersize={8.5in,11in}, margin=1in]{geometry}
\usepackage{float}
\usepackage{caption}
\usepackage{esint}
\usepackage{url}
\usepackage{enumitem}
\usepackage{subfig}
\usepackage{wasysym}
\newcommand{\ilc}{\texttt}
\usepackage{etoolbox}
\usepackage{algorithm}
\usepackage{changepage}
% \usepackage{algorithmic}
\usepackage[noend]{algpseudocode}
\usepackage{tikz}
\usetikzlibrary{matrix,positioning,arrows.meta,arrows}
\patchcmd{\thebibliography}{\section*{\refname}}{}{}{}
% \PassOptionsToPackage{hyphens}{url}\usepackage{hyperref}

\providecommand{\myceil}[1]{\left \lceil #1 \right \rceil }
\providecommand{\myfloor}[1]{\left \lfloor #1 \right \rfloor }


\begin{document}



\title{\textbf{MATH 307: Group Homework 1}}


\author{Shaochen (Henry) ZHONG, \ilc{sxz517} \\
Zhitao (Robert) CHEN, \ilc{zxc325} \\
John MAYS, \ilc{jkm100}}
\date{Due and submitted on 02/05/2021 \\ Spring 2021, Dr. Guo}
\maketitle



\section*{Problem 1}

\subsection*{$f: \mathbb{N} \to \mathbb{Z}$, $f(n) = z$ where $z^2 = n$}

It is not a function as $\exists  n_0 \in \mathbb{N}$, assume $z_0 = \sqrt{n_0}$, we may have $z = \{z_0, -z_0\}$ both satisifying the requirement of $z^2 = n$. This suggests a one-to-many relationship and therefore not a function.


\subsection*{$f: \mathbb{Q} \to \mathbb{Z}$, $f(m/n) = 3m + 2n$}

To prove by contradiction. Assume $\frac{m}{n} = z$, we have $z = \frac{km}{kn}$ for $k \in \mathbb{Q}$; thus for the same input $z$, we will have multiple output values as $3km + 2kn$. This suggests a one-to-many relationship and therefore not a function.

\subsection*{$g: \mathbb{R} \to \mathbb{R}$, $g(x) = x^3$}

It is a function as for every input $x \in \mathbb{R}$, there are only one and only one output value $x^3$. This function is  \textit{bijective} as it shows perfect matching between its domain and codomain -- that every element in its domain is matched to an unique element in its codomain, and vice versa.

\subsection*{$g: \mathbb{R} \to \mathbb{R}$, $g(x) = x^2$}

It is a function as for every input $x \in \mathbb{R}$, there are only one and only one output value $x^2$. This function is not \textit{injective} as for $\{x_0, x_1\} \in \mathbb{R}$, let $x_0 = -x_1$, we have $f(x_0) = f(x_1) = (x_0)^2$; this suggests a many-to-one relationship and therefore not an injective function. This function is also not \textit{surjective} as we have $\mathbb{R}$ being the codomain of the function, but any $\mathbb{Z^-}$ among the codomain can't be matched to any domain in  $\mathbb{R}$. So it is just a general function.

\section*{Problem 2}

We have $f \circ g(x) = f(g(x)) = \sin(x^2)$ and  $g \circ f(x) = g(f(x)) = (\sin x)^2$. It is clear they are both defined, but not equal.

\section*{Problem 3}

To permute the elements of sets, we have $A = \{0, 1\}$ and $B = \{0, \pm1, \pm2, -3, -4, -5, ...\}$. Thus:

\begin{itemize}
    \item $A \cup B = \{0, \pm1, \pm2, -3, -4, -5, ...\} = B$
    \item $A \cap B = \{0, 1\} = A$
    \item $A \backslash B = \{x \mid x \in A, x \not\in B\} = \emptyset$
\end{itemize}

\noindent As every element of $A$ is an element of $B$, we have $A \subset B$.
\end{document}

